\subsection{Combination of FFR and PF schedule}
\label{sec_ffr_pf}
The study in~\cite{cite_docomo3} aims to let cell-edge users have better network experience 
which is similar to the work in~\cite{cite_docomo2}. 
They propose scheduling algorithm extends the concept of fractional frequency reuse (FFR) and 
weighted proportional fair (PF)-based multiuser scheduling to nonorthogonal access with a 
successive interference cancellation (SIC) in the cellular downlink. 

A typical method, soft FFR, the overall system transmission bandwidth is divided into two parts: 
a frequency band with priority given to cell-edge users ($\mathcal{B}_{\text{edge}}$) and that 
with priority given to cell-interior users ($\mathcal{B}_{\text{inner}}$). The bandwidth for the
edge band is assumed to be 1/3 of the overall system transmission bandwidth.

The average user throughput of user k per frequency block is defined as
\begin{equation}
\label{eq_avg_throughput}
T(k;t+1)=\left(1-\frac{1}{t_c}\right)T(k;t)+\frac{1}{t_c}\left(\frac{1}{B}\sum_{b=1}^{B} R^{\text{(sic)}}_b(k;t)\right),
\end{equation}
where $t$ denotes the time index and $t_c$ defines the time defines the time horizon in which we want
to achieve fairness. Obviously, the larger $t_c$, the less stringent
the fairness constraint, and thus longer delays start appearing
between successive transmissions to the same user.

In FFR, the scheduling metric is affected by the frequency block access policy. 
The sets of users categorized into cell-edge and cell-interior user groups are denoted as 
$\mathcal{K}_{\text{edge}}$ and $\mathcal{K}_{\text{inner}}$ respectively. 
By modifying original PF-based multi-user scheduling, the resource access policy applies the
following selecting criteria:
\begin{equation}
\label{eq_modify_PF1}
f_b(S)=\prod_{k \in S} \left( 1+\frac{\alpha_b(k) R^{\text{(sic)}}_b(k|S;t)}{(t_c-1) T^{\gamma}(k;t)} \right) \text{and}
\end{equation}
and the product of the average user throughput among users is maximized by selecting user 
set $\mathcal{S}_b$ by
\begin{equation}
\label{eq_max_PF1}
\mathcal{S}_b = \text{arg}\,\text{max}_s f_b(\mathcal{S})
\end{equation}
where $S$ is the schedule set and $\gamma(0\leq\gamma)$ is the 
weighting factor designed to achieve better user 
fairness.
Here, soft priority access coefficient $\alpha$ denotes whether or not the
users at the edge can access the inner band or interior users can access
the edge band. The coefficient is defined as follows,
\begin{equation}
\label{eq_modify_PF2}
 \alpha_b(k) = \left\{
  \begin{array}{l l}
    \alpha_{\text{edge}}, & b \in \mathcal{B}_{\text{inner}}, k \in \mathcal{K}_{\text{edge}} \\
    \alpha_{\text{inner}}, & b \in \mathcal{B}_{\text{edge}}, k \in \mathcal{K}_{\text{inner}} \\
    1, & \text{otherwise,}
  \end{array} \right.
\end{equation}

To determine the transmission power in this case,
equation~\ref{eq_modify_PF1} is approximated as
\begin{equation}
\label{eq_approx_for_pwr_alloc}
f_b(\mathcal{S})=\underset{k\in\mathcal{S}}{\sum}{\frac{\alpha_b(k)R_b(k|S;t)}{T^\gamma(k;t)}}
\end{equation}
Which is a weighted sum of the instantaneous user throughput.
Therefore, for given candidate scheduling policy,
the metric can be maximized by water-filling power allocation method.

By the scheduling method introduced in previous works, the performance of geometric
mean user throughput of Non-orthogonal access with SIC networks is better than
OFDM networks. However the performance of PF scheduler in NOMA-SIC access networks
is not evaluated and remains unclear.
